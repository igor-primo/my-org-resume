% Created 2024-07-02 ter 17:49
% Intended LaTeX compiler: pdflatex
\documentclass[10pt,a4paper,ragged2e,withhyper]{altacv}

% Change the page layout if you need to
\geometry{left=1.25cm,right=1.25cm,top=1.5cm,bottom=1.5cm,columnsep=1.2cm}

% Use roboto and lato for fonts
\renewcommand{\familydefault}{\sfdefault}

% Change the colours if you want to
\definecolor{SlateGrey}{HTML}{2E2E2E}
\definecolor{LightGrey}{HTML}{666666}
\definecolor{DarkPastelRed}{HTML}{450808}
\definecolor{PastelRed}{HTML}{8F0D0D}
\definecolor{GoldenEarth}{HTML}{E7D192}
\colorlet{name}{black}
\colorlet{tagline}{PastelRed}
\colorlet{heading}{DarkPastelRed}
\colorlet{headingrule}{GoldenEarth}
\colorlet{subheading}{PastelRed}
\colorlet{accent}{PastelRed}
\colorlet{emphasis}{SlateGrey}
\colorlet{body}{LightGrey}

% Change some fonts, if necessary
\renewcommand{\namefont}{\Huge\rmfamily\bfseries}
\renewcommand{\personalinfofont}{\footnotesize}
\renewcommand{\cvsectionfont}{\LARGE\rmfamily\bfseries}
\renewcommand{\cvsubsectionfont}{\large\bfseries}

% Change the bullets for itemize and rating marker
% for cvskill if you want to
\renewcommand{\itemmarker}{{\small\textbullet}}
\renewcommand{\ratingmarker}{\faCircle}

\usepackage[rm]{roboto}
\usepackage[defaultsans]{lato}
\usepackage{paracol}
\columnratio{0.6} % Set the left/right column width ratio to 6:4.
\usepackage[bottom]{footmisc}
\DeclareNameAlias{sortname}{given-family}
\addbibresource{aidan.bib}
\usepackage[sorting=none,sortcites=true,doi=false,url=false,giveninits=true,hyperref]{biblatex}

\author{Igor Primo}
\date{\today}
\title{Org Resume}
\begin{document}

\name{Igor de Souza Carvalhal Primo}
\photoR{2.8cm}{ego.jpeg}
\tagline{DevOps Engineer}
\personalinfo{
  %\homepage{www.aidanscannell.com}
  \email{igorprimo62@gmail.com}
  \phone{+55 (79) 999657773}
  \location{Aracaju, SE}
  \github{igor-primo}
  \linkedin{www.linkedin.com/in/igor-primo}
  \dob{20 Outubro 1999}
  %\driving{UK Driving Licence}
}
\makecvheader
\begin{paracol}{2}
\begin{quote}
 ``Eu transformo suas ideias em software correto e de baixa manutenção, e transformo o estresse do seu processo de desenvolvimento, e de entrega para o usuário, em flow.''
\end{quote}
\cvsection{Experiência}
\label{sec:org3ecf558}
\cvevent{Engenheiro DevOps}{ Tribunal do Trabalho da 20a Região}{ Out 2022 -- Agora}{ Aracaju, SE}

Nessa posição, para enriquecer minhas habilidades de desenvolviemnto, pratiquei os fundamentos de DevOps.

Atuei também como administrador de sistemas.

Minhas realizações:
\begin{itemize}
\item Desenvolvi scrips de Powershell para automação de atividades do WSUS e do Active Directory.
\item Executei processo de hardening em servidores Windows, atualizando e reconfigurando softwares para remover vulnerabilidades ou mitigá-las.
\item Elaborei e apresentei para Divisão de Infraestrutura uma política de Gestão de Segurança de Informação com elaboração de processos que aumentem e mantenham o nível de endurecimento dos servidores.
\item Instalei um cluster Kubernetes utilizando Ansible.
\item Instalei e configurei instâncias de softwares para homologação no novo ambiente Kubernetes, como Gitlab, Grafana e Prometheus.
\item Automatizei o provisionamento de servidores através de Terraform em um nó Proxmox on-premises.
\item Elaborei a documentação dos processos de instalação e dos arquivos de configuração desses softwares para disseminação de conhecimento entre os membros da equipe.
\item Construí e testei pipelines CI/CD.
\end{itemize}

\cvtag{Kubernetes}
\cvtag{Helm}
\cvtag{Linux Administration}
\cvtag{Ansible}
\cvtag{ITIL}
\cvtag{Shell}
\cvtag{Virtualization}
\cvtag{Grafana}
\cvtag{Prometheus}
\cvtag{CI/CD}

\par\divider
\cvevent{Desenvolvedor Back-End}{ Universidade Federal de Sergipe}{ Jan 2022 -- Jan 2023}{ Aracaju, SE}

Atuei como desenvolvedor back-end.

Minhas realizações:
\begin{itemize}
\item Desenvolvi a API RESTful para uma aplicação que possibilita a comunicação rápida entre as equipes laboratorial e assistencial de um hospital, permitindo a troca de informações a respeito de resultados de exames e a devida dosagem de antibióticos.
\item Configurei banco de dados PostgreSQL e gerenciei dados da aplicação, realizando migrações e modificações adequadas quando necessárias.
\end{itemize}

\cvtag{Node.js}
\cvtag{Docker}
\cvtag{Git}
\cvtag{PostgreSQL}
\cvtag{Netlify}
\cvtag{Render}
\cvtag{Metodologias Ágeis}

\newpage

\par\divider
\cvevent{Desenvolvedor Back-End}{ Universidade Federal de Sergipe}{ Set 2021 -- Ago 2022}{ Aracaju, SE}

Atuei como desenvolvedor back-end. Minhas realizações:
\begin{itemize}
\item Desenvolvi uma API RESTful em Node.js como componente para uma aplicação de fomento ao consumo consciente.
\item Configurei banco de dados PostgreSQL e gerenciou dados da aplicação, realizando migrações e modificações adequadas quando necessárias.
\item Realizei análise de requisitos e de conformidade do software para com os seus requisitos.
\end{itemize}

\cvtag{Node.js}
\cvtag{Docker}
\cvtag{Git}
\cvtag{PostgreSQL}
\cvtag{Metodologias Ágeis}
\newpage
\switchcolumn
\cvsection{Projetos}
\label{sec:orgefac9da}
\cvevent{Petri}{ Universidade Federal de Sergipe}{ Jan 2022 - Jan 2023}{ Aracaju, SE}

Sistema web e mobile que busca encurtar o tempo de comunicação entre a equipe laboratorial e a equipe assistencial do Hospital Universitário de Sergipe, aumentando a eficácia no tratamento antibiótico.

A aplicação está em \url{https://petrisaude.netlify.app/signin}

\cvtag{Node.js}
\cvtag{Docker}
\cvtag{Git}
\cvtag{PostgreSQL}
\cvtag{Metodologias Ágeis}
\cvtag{Trabalho em Equipe}
\cvtag{Levantamento de Requisitos}

\divider
\cvsection{Habilidades}
\label{sec:org38971dd}

\cvtag{Node.js}
\cvtag{JavaScript}
\cvtag{PureScript}
\cvtag{TypeScript}
\cvtag{React}
\cvtag{React-Native}
\cvtag{MongoDB}
\cvtag{PostgreSQL}
\cvtag{Java}
\cvtag{C}

\divider

\cvtag{DevOps}
\cvtag{Linux Administration}
\cvtag{Kubernetes}
\cvtag{Helm}
\cvtag{AWS}
\cvtag{Terraform}

\divider

\cvtag{Git/GitHub}
\cvtag{Emacs}
\cvtag{VIM}

\divider

\cvtag{OpenBSD}
\cvtag{Haskell}
\cvtag{Obelisk}
\cvtag{IHP}

\divider
\cvsection{Formação}
\label{sec:org3a2df5e}

\cvevent{Engenharia de Computação}{ Universidade Federal de Sergipe}{ 2018 - Agora}{}

\cvevent{Curso de Inglês Avançado}{ Yázigi}{ 2016 - 2018}{}
\newpage
\cvsection{Línguas}

\cvskill{Português}{5}
\divider

\cvskill{Inglês}{5}
\divider

\cvskill{Espanhol}{3}
\divider

% %% Yeah I didn't spend too much time making all the
% %% spacing consistent... sorry. Use \smallskip, \medskip,
% %% \bigskip, \vpsace etc to make ajustments.
% \medskip

\newpage
\end{paracol}
\end{document}
\end{document}
