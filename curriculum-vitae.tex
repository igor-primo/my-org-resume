% Created 2024-06-27 qui 12:36
% Intended LaTeX compiler: pdflatex
\documentclass[10pt,a4paper,ragged2e,withhyper]{altacv}

% Change the page layout if you need to
\geometry{left=1.25cm,right=1.25cm,top=1.5cm,bottom=1.5cm,columnsep=1.2cm}

% Use roboto and lato for fonts
\renewcommand{\familydefault}{\sfdefault}

% Change the colours if you want to
\definecolor{SlateGrey}{HTML}{2E2E2E}
\definecolor{LightGrey}{HTML}{666666}
\definecolor{DarkPastelRed}{HTML}{450808}
\definecolor{PastelRed}{HTML}{8F0D0D}
\definecolor{GoldenEarth}{HTML}{E7D192}
\colorlet{name}{black}
\colorlet{tagline}{PastelRed}
\colorlet{heading}{DarkPastelRed}
\colorlet{headingrule}{GoldenEarth}
\colorlet{subheading}{PastelRed}
\colorlet{accent}{PastelRed}
\colorlet{emphasis}{SlateGrey}
\colorlet{body}{LightGrey}

% Change some fonts, if necessary
\renewcommand{\namefont}{\Huge\rmfamily\bfseries}
\renewcommand{\personalinfofont}{\footnotesize}
\renewcommand{\cvsectionfont}{\LARGE\rmfamily\bfseries}
\renewcommand{\cvsubsectionfont}{\large\bfseries}

% Change the bullets for itemize and rating marker
% for cvskill if you want to
\renewcommand{\itemmarker}{{\small\textbullet}}
\renewcommand{\ratingmarker}{\faCircle}

\usepackage[rm]{roboto}
\usepackage[defaultsans]{lato}
\usepackage{paracol}
\columnratio{0.6} % Set the left/right column width ratio to 6:4.
\usepackage[bottom]{footmisc}
\DeclareNameAlias{sortname}{given-family}
\addbibresource{aidan.bib}
\usepackage[sorting=none,sortcites=true,doi=false,url=false,giveninits=true,hyperref]{biblatex}

\author{Aidan Scannell}
\date{\today}
\title{Org Resume}
\begin{document}


\name{Igor de Souza Carvalhal Primo}
\photoR{2.8cm}{aidan_portrait.jpeg}
\tagline{Web Developer}

\personalinfo{
  %\homepage{www.aidanscannell.com}
  \email{igorprimo62@gmail.com}
  \phone{+55 (79) 999657773}
  \location{Aracaju, SE}
  \github{igor-primo}
  \linkedin{www.linkedin.com/in/igor-primo}
  \dob{20 Outubro 1999}
  %\driving{UK Driving Licence}
}
\makecvheader

\begin{paracol}{2}

\begin{quote}
 ``Desenvolvimento é uma corrida contra o universo: desenvolvedores fazem programas idiot-proof, o universo produz idiotas (Rick Cook). Meus programas idiot-proof são normalmente escritos em React ou Node.js. Mas nenhuma curva de aprendizado é íngrime para eternos estudantes.''
\end{quote}
\cvsection{Experiência}
\label{sec:orgb4db928}
\cvevent{Engenheiro DevOps}{ Tribunal do Trabalho da 20a Região}{ Out 2022 -- Agora}{ Aracaju, SE}

Nessa posição, pratiquei os fundamentos da cultura DevOps, enriquecendo minhas habilidades de desenvolvimento.
Aprendi a configurar e utilizar pipelines CI/CD com o Gitlab.

Atuei também como administrador de sistemas.

Minhas realizações:
\begin{itemize}
\item Instalação de um cluster Kubernetes utilizando Ansible.
\item Instalação e configuração de instâncias de softwares para homologação no novo ambiente Kubernetes, como
Gitlab, Grafana e Prometheus.
\item Provisionamento automatizado de servidores através de Terraform em um nó Proxmox on-premises.
\item Elaboração de documentação dos processos de instalação e dos arquivos de configuração desses softwares
para disseminação de conhecimento entre os membros da equipe.
\end{itemize}

\cvtag{Kubernetes}
\cvtag{Linux Administration}
\cvtag{Ansible}
\cvtag{ITIL}
\cvtag{Shell}
\cvtag{Virtualization}
\par\divider
\cvevent{Desenvolvedor}{ Universidade Federal de Sergipe}{ Set 2021 -- Ago 2022}{ Aracaju, SE}

Atuou como desenvolvedor back-end. Realizações:
\begin{itemize}
\item Desenvolveu uma API RESTful em Node.js como componente para uma aplicação de fomento ao consumo consciente.
\item Configurou banco de dados PostgreSQL e gerenciou dados da aplicação, realizando migrações e modificações adequadas quando necessárias.
\item Realizou análise de requisitos e de conformidade do software para com os seus requisitos.
\end{itemize}

\cvtag{Node.js}
\cvtag{Docker}
\cvtag{Git}
\cvtag{PostgreSQL}
\cvtag{Metodologias Ágeis}

\newpage

\switchcolumn

\cvsection{Projetos}
\label{sec:orgfa011a6}
\cvevent{Petri}{ Universidade Federal de Sergipe}{ Jan 2022 - Jan 2023}{ Aracaju, SE}

Sistema web e mobile que busca encurtar o tempo de comunicação entre a equipe laboratorial e a equipe assistencial do Hospital Universitário de Sergipe, aumentando a eficácia no tratamento antibiótico.

A aplicação está em \url{https://petrisaude.netlify.app/signin}

\cvtag{Node.js}
\cvtag{Docker}
\cvtag{Git}
\cvtag{PostgreSQL}
\cvtag{Metodologias Ágeis}
\cvtag{Trabalho em Equipe}
\cvtag{Levantamento de Requisitos}

\cvsection{Habilidades}
\label{sec:org23d2fa3}

\cvtag{Node.js}
\cvtag{React}
\cvtag{React-Native}
\cvtag{MongoDB}
\cvtag{PostgreSQL}
\cvtag{Java}
\cvtag{C}

\divider

\cvtag{DevOps}
\cvtag{Linux Administration}
\cvtag{Kubernetes}
\cvtag{AWS}

\divider

\cvtag{Git/GitHub}
\cvtag{Emacs}
\cvtag{VIM}

\divider

\cvtag{OpenBSD}
\cvtag{Haskell}
\cvtag{Obelisk}
\cvtag{IHP}

\cvsection{Formação}
\label{sec:org3d2afc6}

\cvevent{Ciência de Computação}{ Universidade Federal de Sergipe}{ 2018 - Agora}{}

\newpage

\end{paracol}
\end{document}
\end{document}
